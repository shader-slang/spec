\Chapter{Statements}{stmt}

Statements are used to define the bodies of functions and deterine order of evaluation and control flow for an entire program.
Statements are distinct from expressions in that statements do not yield results and do not have types.

This section lists the kinds of statements supported by Slang.

\Section{Expression Statement}{expr}

An expression statement consists of an expression followed by a semicolon:

\begin{codeblock}
doSomething();
a[10] = b + 1;
\end{codeblock}

An implementation may warn on an expression statement that has to effect on the results of execution.

\Section{Declaration Statement}{decl}

A declaration may be used as a statement:

\begin{codeblock}
let x = 10;
var y = x + 1;
int z = y - x;
\end{codeblock}

\begin{Note}
Currently only variable declarations are allowed in statement contexts, but other kinds of declarations may be enabled in the future.
\end{Note}

\Section{Block Statement}{block}

A block statement consists of zero or more statements wrapped in curly braces \Char{\{}\Char{\}}:

\begin{codeblock}
{
	int x = 10;
	doSomething(x);
}
\end{codeblock}

A block statement provides local scoping to declarations.
Declarations in a block are visible to later statements in the same block, but not to statements or expressions outside of the block.

\Section{Empty Statement}{empty}

A single semicolon (\Char{;}) may be used as an empty statement equivalent to an empty block statement \Char{\{}\Char{\}}.

\Section{Conditional Statements}{cond}

\SubSection{If Statement}{if}


An \SpecDef{if statement} consists of the \code{if} keyword and a conditional expression in parentheses, followed by a statement to execute if the condition is true:

\begin{codeblock}
if(somethingShouldHappen)
    doSomething();
\end{codeblock}

An if statement may optionally include an \SpecDef{else clause} consisting of the keyword \code{else} followed by a statement to execute if the condition is false:

\begin{codeblock}
if(somethingShouldHappen)
 	doSomething();
else
	doNothing();
\end{codeblock}

\SubSection{Switch Statement}{switch}

A \SpecDef{switch statement} consists of the \code{switch} keyword followed by an expression wrapped in parentheses and a \SpecDef{body statement}:

\begin{codeblock}
switch(someValue)
{
	...
}
\end{codeblock}

The body of a switch statement must be a block statement, and its body must consist of switch case clauses.
A \SpecDef{switch case clause} consists of one or more case labels or default labels, followed by one or more statements:

\begin{codeblock}
// this is a switch case clause
case 0:
case 1:
    doBasicThing();
    break;

// this is another switch case clause
default:
    doAnotherThing();
    break;
\end{codeblock}

A \SpecDef{case label} consists of the keyword \code{case} followed by an expresison and a colon (\Char{:}).
The expression must evaluate to a compile-time constant integer.

A \SpecDef{default label} consists of the keyword \code{default} followed by a colon (\Char{:}).

It is an error for a case label or default label to appear anywhere other than the body of a \code{switch} statement.
It is an error for a statement to appear inside the body of a \code{switch} statement that is no part of a switch case clause.

Each switch case clause must exit the \code{switch} statement via a \code{break} or other control transfer statement.
"Fall-through" from one switch case clause to another is not allowed.

\Section{Loop Statements}{loop}

\SubSection{For Statement}{for}

A \SpecDef{for statement} uses the following form:

\begin{codeblock}
for( <initial statement> ; <condition expression> ; <side effect expression> ) <body statement>
\end{codeblock}

The \SpecDef{initial statement} is optional, but may declare a variable whose scope is limited to the for statement.

The \SpecDef{condition expression} is optional. If present it must be an expression that can be coerced to type \code{bool}. If absent, a true value is used as the condition.

The \SpecDef{side effect expression} is optional. If present it will executed for its effects before each testing the condition for every loop iteration after the first.

The \SpecDef{body statement} is a statement that will be executed for each iteration of the loop.

\SubSection{While Statement}{while}

A \SpecDef{while statement} uses the following form:

\begin{codeblock}
while( <condition expression> ) <body statement>
\end{codeblock}

and is equivalent to a \code{for} loop of the form:

\begin{codeblock}
for( ; <condition expression> ; ) <body statement>
\end{codeblock}

\SubSection{Do-While Statement}{dowhile}

A \SpecDef{do-while statement} uses the following form:

\begin{codeblock}
do <body statement> while( <condition expression> )
\end{codeblock}

and is equivalent to a \code{for} loop of the form:

\begin{codeblock}
for(;;)
{
	<body statement>
	if(<condition expression>) continue; else break;
}
\end{codeblock}

\Section{Control Transfer Statements}{control}

\SubSection{Break Statement}{break}

A \code{break} statement transfers control to after the end of the closest lexically enclosing switch statement or loop statement:

\begin{codeblock}
break;
\end{codeblock}

\SubSection{Continue Statement}{continue}

A \code{continue} statement transfers control to the start of the next iteration of a loop statement.
In a for statement with a side effect expression, the side effect expression is evaluated when \code{continue} is used:

\begin{codeblock}
break; 
\end{codeblock}

\SubSection{Return Statement}{continue}

A \code{return} statement transfers control out of the current function.

In the body of a function with a \code{void} result type, the \code{return} keyword may be followed immediately by a semicolon:

\begin{codeblock}
return;
\end{codeblock}

Otherwise, the \code{return} keyword must be followed by an expression to use as the value to return to the caller:

\begin{codeblock}
return someValue;
\end{codeblock}

The value returned must be able to coerce to the result type of the lexically enclosing function.

\SubSection{Discard Statement}{discard}

A \code{discard} statement can only be used in the context of a fragment shader, in which case it causes the current invocation to terminate and the graphics system to discard the corresponding fragment so that it does not get combined with the framebuffer pixel at its coordinates.

Operations with side effects that were executed by the invocation before a \code{discard} will still be performed and their results will become visible according to the rules of the platform.

\Section{Compile-Time For Statement}{for.compile-time}

A \SpecDef{compile-time for statement} is used as an alternative to preprocessor techniques for loop unrolling.
It looks like:

\begin{codeblock}
$\mbox{\textdollar}$for( <name> in Range(<initial-value>, <upper-bound>)) <body statement>
\end{codeblock}

The \SpecDef{initial value} and \SpecDef{upper bound} expressions must be compile-time constant integers.
The semantics of a compile-time for statement are as if it were expanded into:

\begin{codeblock}
{
	let <name> = <initial-value>;
	<body statement>
}
{
	let <name> = <initial-value> + 1;
	<body statement>
}
...
{
	let <name> = <upper-bound> - 1;
	<body statement>
}
\end{codeblock}

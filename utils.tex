% utils.tex
%
% This file defines all the utility commands that are needed to prepare the rest of the document.


% \DefineVariable : a helper command for defining a getter and setter pair.
%
% Doing `\DefineVariable{Thing}` will allow you to use `\SetThing{...}` to
% set a variable and `\Thing` to access its current value.
%
%\makeatletter
%\newcommand*\DefineVariable[1]{\@namedef{Set#1}##1{\global\@namedef{#1}{##1}}}
%\makeatother
%
%\DefineVariable{ChapterLabel}
%\DefineVariable{SectionLabel}
%\DefineVariable{SubSectionLabel}
%
%\newcommand*{ChapterPath}{\ChapterLabel}
%\newcommand*{SectionPath}{\ChapterPath.\SectionLabel}
%\newcommand*{SubSectionPath}{\SectionPath.\SubSectionLabel}
%
%\setcounter{secnumdepth}{3}
%
%\newcommand*{\Chapter}[2]{\SetChapterLabel{#2}\chapter[#1\hfill[\ChapterPath]]\label{\ChapterPath}}
%\newcommand*{\Section}[2]{\SetSectionLabel{#2}\section[#1\hfill[\SectionPath]]\label{\SectionPath}}
%\newcommand*{\SubSection}[2]{\SetSubSectionLabel{#2}\subsection[#1\hfill[\SubSectionPath]]\label{\SubSectionPath}}
%
\newenvironment{Callout}[2]
    {
    \begin{center}
    \begin{tabular}{p{0.9\textwidth}}
    \cellcolor{#2}
    \textbf{#1}: \\
    }
    {
    \\\\
    \end{tabular}
    \end{center}
    }

\newcommand{\DefineCallout}[3]{\newenvironment{#1}{\begin{Callout}{#2}{#3}}{\end{Callout}}}

\definecolor{NoteColor}{gray}{0.85}
\definecolor{IncompleteColor}{gray}{0.85}
\definecolor{SyntaxColor}{gray}{0.85}

\DefineCallout{Note}{Note}{NoteColor}
\DefineCallout{Incomplete}{Incomplete}{IncompleteColor}
\DefineCallout{Syntax}{Syntax}{SyntaxColor}
\DefineCallout{Lexical}{Syntax}{SyntaxColor}



%
%\definecolor{ListingKeywordColor}{gray}{0.5}
%\definecolor{ListingTypeColor}{gray}{0.5}
%
%\lstdefinelanguage{slang}{
%    classoffset=0,morekeywords={class,func,let,struct},\keywordstyle=\color{ListingKeywordColor},
%    classoffset=1,morekeywords={void,int,float,bool},\keywordstyle=\color{ListingTypeColor},
%    classoffset=0,
%    morecomment=[l]{//},
%    morecomment=[s]{/*}{*/},
%    morestring=[b]",
%}
%
%\lstnew

\newcommand*{\Chapter}[2]{\chapter{#1}}
\newcommand*{\Section}[2]{\section{#1}}
\newcommand*{\SubSection}[2]{\subsection{#1}}
\newcommand*{\SubSubSection}[2]{\subsubsection{#1}}



\definecolor{frequency_color}{RGB}{0,176,80}
\definecolor{keyword_color}{RGB}{0,32,96}
\definecolor{comment_color}{RGB}{192,0,0}
\definecolor{type_color}{RGB}{0,112,192}
\definecolor{keyword_arg_color}{RGB}{144,155,170}

\lstdefinestyle{SlangCodeStyle}{
    basicstyle=\mdseries\ttfamily,
    classoffset=0,
    morekeywords={record,concrete,pass,override,virtual,input,abstract,void,mixin,primary,struct,implicit,return,where,shader,class,extends,output,true,false,extend,space,type},
    keywordstyle=\color{keyword_color}\bfseries\ttfamily,
    classoffset=1,
    morekeywords={@Constant,@Uniform,@Vertex,@RasterVertex,@Instance,@Fragment,@Pixel,@E,@Patch,@ControlPoint,@ShadedVertex,@AssembledVertex,@RasterVertex,@DomainVertex,@InputControlPoint,@OutputControlPoint,@InputPatch,@OutputPatch,@Sample,@Element,@CoarseVertex,@FineVertex,@Light,@R,@GeometryOutput},
    keywordstyle=\color{frequency_color}\ttfamily,
    classoffset=2,
    morekeywords={Constant,Uniform,Vertex,RasterVertex,Fragment,Pixel,float,int,float4,float3,float2,float4x4,
        VertexBuffer,SimpleTransform,D3D9DrawPass,D3D11DrawPass,ShadedVertex,AssembledVertex,E,Array,Patch,
        ControlPoint,D3D11NullTessellator,D3D11NullGeometryShader,VertexColors,SimpleTransformAndVertexColors,
        DomainVertex,InputControlPoint,OutputControlPoint,InputPatch,OutputPatch,ExtCoarseVertex,
        D3D11TessellatorPassThrough,D3D11GeometryShaderPassThrough,Displacement,OutputStream,T,U,
        ID3D11RenderTargetView,Desc,Linear,SimpleTessellation,PointSprites,
        Color,CoarseVertex,FineVertex,RasterVertex,Simple,VertexStream,
        Base,SkeletalAnimation,MorphTargetAnimation,CubicGregory,CubicGregoryTris,SurfaceAttributes,PackGBuffer,UnpackGBuffer,GenerateSurfaceAttributes,Light,
CubicGregoryQuads,Tris,Quads,DisplacementMapping,RenderToCubeMap,PointSprites,Diffuse,Texturing,
        Phong,EnvMap,PointLight,SpotLight,Transform,Displace,Texture,PNuv,Puv,Texture2D,SamplerState,uint,Composed,Fetch,DirectionalLight,
        PhongMaterial,EnvMapMaterial,NormalMaterial,D3D11NullTessellation,D3D11Tessellation,D3D11GeometryShader,D3D11TriTessellation,D3D11QuadTessellation,Example,Shade,Point,Normal,Derived,Float,vector,R,Graph,Node,Edge,ColorGraph,WeightedGraph,WeightedColorGraph,A,B,D,VertexTrait_Base,VertexTrait_Derived,Derived_Final,Vector,Transform,Transformation,ColorMixin,Tessellate,float3x3},
    keywordstyle=\color{type_color}\ttfamily,
    classoffset=0,
    moredelim=[is][\color{keyword_arg_color}\ttfamily]{|}{|},
    moredelim=[is][\color{red}\ttfamily]{^}{^},
    morecomment=[l]//,
    commentstyle=\color{comment_color},
    tabsize=2,
    mathescape=true}

\lstnewenvironment{codeblock}[1][]
    {\lstset{
        style=SlangCodeStyle,
        basicstyle=\mdseries\ttfamily\small
        #1}}
    {}

\newcommand{\code}[1]{\text{\lstinline[style=SlangCodeStyle]{#1}}}
\newcommand{\kw}[1]{\text{\texttt{\textbf{\textcolor{keyword_color}{#1}}}}}

\newcommand{\SpecDefine}[1]{\text{\emph{#1}}}
\newcommand{\SpecDef}[1]{\text{\emph{#1}}}

\newcommand{\SynDefine}[1]{\text{\emph{#1} ::=}}
\newcommand{\SynRef}[1]{\text{\emph{#1}}}
\newcommand{\SynOr}{\text{ $\vert$ }}
\newcommand{\SynStar}{\text{*}}
\newcommand{\SynOpt}{\text{?}}

\newcommand{\MetaVar}[1]{\textbf{#1}}

\newcommand{\Char}[1]{\code{#1}}
\Chapter{Preprocessor}{prepro}

As the last phase of lexical processing the token sequence of a source unit is preprocessed to produce a new token stream.

The preprocessor supported by Slang is derived from the C/C++ preprocessor with a few changes and extensions.

\begin{Incomplete}
We \emph{either} need to pick a normative reference here for some existing preprocessor (and thus bind ourselves to eventually supporting the semantics of that normative reference), \emph{or} we need to take the time to fully document what the semantics of our current preprocessor are.
\end{Incomplete}

Slang programs may use the following preprocessor directives, with the same semantics as their C/C++ equivalent:

\begin{itemize}
\item \code{#include}
\item \code{#define}
\item \code{#undef}
\item \code{#if}, \code{#ifdef}, \code{#ifndef}
\item \code{#else}, \code{#elif}
\item \code{#endif}
\item \code{#error}
\item \code{#warning}
\item \code{#line}
\item \code{#pragma}
\end{itemize}

An implementation may use any implementation-specified means to resolve paths provided to \code{#include} directives.

An implementation that supports \code{#pragma  once} may use any implementation-specified means to determine if two source units are identical.

\Section{Changes}{changes}

The input to the Slang preprocessor is a token sequence produced by the rules in Chapter 2, and does not use the definition of "preprocessor tokens" as they are used by the C/C++ preprocessor.

\begin{Note}
The key place where this distinction matters is in macros that perform token pasting.
The input to the Slang preprocessor must be a valid sequence of tokens *before* any token pasting occurs.
\end{Note}

When tokens are pasted with the \code{##} operator, the resulting concatenated text is decomposed into one or more new tokens.
It is an error if the concatenated text does not form a valid sequence of tokens.

\begin{Note}
The C/C++ preprocessor always yields a single token from any token pasting, whether or not that token is valid.
\end{Note}

\Section{Extensions}{extensions}

\begin{Incomplete}
At the very least we need to document support for the \code{#version} directive, if we intend to keep it.
\end{Incomplete}

